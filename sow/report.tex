% SOW must be emailed to the course instructor at akhamis@pami.uwaterloo.ca
% with subject SOW-[Your Project number] no later than June 9, 2014.

% (We are team #27)

\documentclass[a4paper]{article}

\usepackage[english]{babel}
\usepackage[utf8]{inputenc}
\usepackage{amsmath}
\usepackage{graphicx}
\usepackage{parskip}
\usepackage{amssymb}

\title{Statement of Work}
\author{
	Group 27 \\
	\\
	Lucas Wojciechowksi, Ariel Weingarten, Alexander Maguire, \\
	Austin Dobrik, Dane Carr}
\date{\today}

\begin{document}

\maketitle

\section{Problem Characterization}

% This project should start by studying comprehensively the problem in order
% to characterize its complexity and its main challenges. Collect related
% resources and conduct a critical survey on similar solutions reported in the
% literature.

% Complex Task Allocation Problem: In mobile surveillance systems, complex
% task allocation addresses how to optimally assign a set of surveillance
% tasks to a set of mobile sensing agents to maximize overall expected
% performance, taking into account the priorities of the tasks and the skill
% ratings of the mobile sensors.

% Suggested reference papers:
%
%  - Alaa Khamis, Ahmed Elmogy and Fakhreddine Karray, "Complex Task Allocation
%    in Mobile Surveillance Systems," Journal of Intelligent and Robotic Systems,
%    Springer
%  - Mohamed Badreldin, Ahmed Hussein and Alaa Khamis, "A Comparative Study
%    between Optimization and Market-based Approaches to Multi-robot Task
%    Allocation," Advances in Artificial Intelligence Journal, 2013.
%  - Alaa Khamis, Cooperative Multirobot Systems, Plenary Talk at IAC2014].

\section{Problem Formulation}

% In this step, an initial statement of the problem should be made. The
% internal and external factors and the objective(s) of the problem must be
% outlined. Describe the problem as standard optimization problem as
% following:
%
% Find
% $$X = {x_1, x_2, ..., x_n}^T$$
% which minimizes/maximizes $f(X)$ subject to equality constraints
% $$\forall j \in [1, m], l_j(X) = 0$$
% and inequality constraints
% $$\forall j \in [1, p], g_j(X) \leq 0$$
%
% Define the decision variable or design vector, objective function/s and the
% problem constraints.

\subsection{Givens}

Suppose you have a number of mobile sensors
$$\mathit{sensors} = \{ \mathit{sensor}_i | i \in [1, n_\mathit{sensors}] \}$$

Using these mobile sensors, you are to peform a number of tasks
$$\mathit{tasks} = \{ \mathit{task}_i | i \in [1, n_\mathit{tasks}] \}$$

There is one exactly one task per sensor
$$n = n_\mathit{sensors} = n_\mathit{tasks}$$

Each task has an assigned ``priority"
% $$\forall \mathit{task_i} \in \mathit{tasks}, \mathit{priority}(\mathit{task}_i) \in [0 ,1]$$
$$\mathit{priority}(\mathit{task}_i) \in [0 ,1]$$

Sensors are not homogeneous; some are more ``skilled" at certain tasks than others
% $$\forall \mathit{sensor_i} \in \mathit{sensors}, \forall \mathit{task_j} \in \mathit{tasks}, \mathit{skill}(\mathit{sensor}_i, \mathit{task}_j) \in [0 ,1]$$
$$\mathit{skill}(\mathit{sensor}_i, \mathit{task}_j) \in [0 ,1]$$

\subsection{Problem}

Our objective is to find an injective mapping of $\mathit{task}$s to $\mathit{sensor}$s that optimizes the overall expected performance. This mapping can be expressed as a vector
$$S = [s_1, ..., s_n]^T$$
$$s_i = j \Rightarrow \mathit{sensor}_i \text{ performs } \mathit{task}_j$$

Our objective is to maximize our objective function, $\mathit{performance}(S)$
$$\mathit{performance}(S) = \sum_{i=1}^n \mathit{skill}(\mathit{sensor}_i, \mathit{task}_{s_i}) \cdot \mathit{priority}(\mathit{task}_{s_i})$$


\section{Problem Modeling}

% In this important step, an abstract mathematical model is built for the
% problem. The modeler can be inspired by similar models in the literature.
% This will reduce the problem to well-studied optimization models. Usually,
% models we are solving are simplifications of the reality. They involve
% approximations and sometimes they skip processes that are complex to
% represent in a mathematical model. For example, travelling salesman problem
% (TSP) and multiple travelling salesman problem (mTSP) are commonly used as
% models for real-life problems such as school bus routing problem and complex
% task allocation problem.

The complex task allocation problem can be modeled as a linear assignment problem.  The solution can be represented as a bipartite graph $G=(S,T,E)$.  The vertex sets $S$ and $T$ each have $n$ vertices, and $E$ contains $n$ edges. $x_{i,j}$ represents an edge between $S_i$ and $T_j$. 

A solution can be represented as a matrix $X_G$, where 

$$x_{i,j} = \left\{
     \begin{array}{lr}
       1 & : x \in E\\
       0 & : x \notin E
     \end{array}
   \right.$$

Let $p(t_i)$ equal the priority of task $i$.
Let $s(s_j, t_i)$ equal the skill of sensor $j$ at task $i$. The performance can be represented as a matrix

$P = \begin{bmatrix}
p(t_0) \cdot s(s_0, t_0) & \cdots & p(t_n) \cdot s(s_0, t_n) \\
\vdots & \ddots & \vdots \\
p(t_0) \cdot s(s_n, t_0) & \cdots & p(t_n) \cdot s(s_n, t_n) \\
\end{bmatrix}$

The performance of a solution can be calculated by multipling a solution matrix and the performance matrix, then summing the elements of the resulting matrix.

E.g.
$$X = \begin{bmatrix}
0 & 1 & 0 \\
1 & 0 & 0 \\
0 & 0 & 1
\end{bmatrix}$$
$$P = \begin{bmatrix}
0.2 & 0.4 & 0.1 \\
0.6 & 0.8 & 0.9 \\
0.3 & 0.2 & 0.5
\end{bmatrix}$$

$$XP = 
\begin{bmatrix}
0 & 1 & 0 \\
1 & 0 & 0 \\
0 & 0 & 1
\end{bmatrix}
\begin{bmatrix}
0.2 & 0.4 & 0.1 \\
0.6 & 0.8 & 0.9 \\
0.3 & 0.2 & 0.5
\end{bmatrix} = 
\begin{bmatrix}
0 & 0.4 & 0 \\
0.6 & 0 & 0 \\
0 & 0 & 0.5
\end{bmatrix}$$

$$performance(X) = \sum_{i=1}^{3}\sum_{j=1}^{3}XP_{i,j} = 1.5$$

\section{Reduced Size Problem}

% Describe a reduced version of the problem in order to be used to perform
% hand-iterations. For example, if you are studying plant layout problem
% (PLP), a reduced size problem may contain only 4 departments (facilities)
% and 4 locations.

\section{Real Problem}

% The real size problem formulation and modeling will be used to show the
% ability of the proposed solutions implemented in handling large size
% problems.


\end{document}