\documentclass[a4paper]{article}

% \usepackage[english]{babel}
% \usepackage[utf8]{inputenc}
% \usepackage{amsmath}
% \usepackage{graphicx}
% \usepackage[colorinlistoftodos]{todonotes}
% \documentclass[a4paper]{article}

\usepackage[english]{babel}
\usepackage[utf8]{inputenc}
\usepackage{amsmath}
\usepackage{graphicx}
\usepackage{parskip}
\usepackage{amssymb}

\newcommand{\coinblack}{\blacksquare}
\newcommand{\coinwhite}{\square}
\newcommand{\coinempty}{-}

\title{ECE457a Assignment 1}

\author{Lucas Wojciechowksi, Ariel Weingarten, Alexander Macguire, Austin Dobrik, Dane Carr}

\date{\today}

\begin{document}
\maketitle

\section{Question 1}

\subsection{Part A}

The game board consists of 7 squares, indexed from 0 to 6.

We can define every state within the game as a 7-tuple containing
\begin{itemize}
\item 3 black coins, denoted $\coinblack$
\item 2 white coins, denoted $\coinwhite$
\item 2 emptys, denoted $\coinempty$
\end{itemize}
The inital state is $s_0 = (\coinempty \coinempty \coinblack \coinwhite \coinblack \coinwhite \coinblack)$

Coins are moved as adjacent pairs. We define a pair of coins as $p_i$, where $i$ is the lowest-indexed square on which the coins rest. In the initial state, the first pair of coins is $p_2$.

Empty squares on the board always exist as a single adjacent pair, which we will also define by its lowest-indexed square. On the initial game board, the empty pair is at $p_0$.

Lets define a function, $\text{empty\_pair}(s)$ which, given a state $s$, returns the empty pair on the board. For example. $\text{empty\_pair}(s_0) = p_0$.

In any game state, $s$, the valid pairs are
$$\{p_i | i \in [0, 5] \backslash [\text{empty\_pair}(s) - 1, \text{empty\_pair}(s) + 1]\}$$
Each of these pairs corresponds with exactly one valid move
$$p_i \rightarrow \text{empty\_pair}(s)$$



\subsection{Part B}

\subsubsection{Breadth First Search}

\subsubsection{Depth First Search (with depth limit of 3)}

\subsubsection{Hill Climbing Search}

\subsubsection{Best First Search (using heuristic developed in previous answer}

\subsubsection{Compare the performance of the above search strategies in terms of the number of nodes generated}

\section{Question 2}

% Insert picture of the MIN MAX tree from assignment


\subsection{Part A}
8
\begin{center}
\includegraphics[width=1\textwidth]{a1q2a.png}
\end{center}

\subsection{Part B}
Move D

\subsection{Part C}
{O, Q, U, Y}

\subsection{Part D}
Swap B and D, giving a final order of {D C B} on the second level of the tree.
This results in ten nodes being pruned: {Y, I, T, U, F, N, O, G, P, Q}.


\end{document}