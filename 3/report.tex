% (Written responses must be typeset and in pdf. Only one member of the group
% (should submit to the appropriate dropbox on Learn.)

% Due Wed, June 30, 11:59:59 PM

\documentclass[a4paper]{article}

\usepackage[english]{babel}
\usepackage[utf8]{inputenc}
\usepackage{amsmath}
\usepackage{graphicx}
\usepackage{parskip}
\usepackage{amssymb}
\usepackage{mathtools}
\usepackage{svg}

\DeclarePairedDelimiter\ceil{\lceil}{\rceil}
\DeclarePairedDelimiter\floor{\lfloor}{\rfloor}

\title{ECE457A Assignment 2}
\author{
  Group 27 \\
  \\
  Lucas Wojciechowksi, Ariel Weingarten, Alexander Maguire, \\
  Austin Dobrik, Dane Carr}
\date{\today}

\begin{document}

\maketitle

\section{Question 1}

\subsection{Binary Encoding}
\subsection{Part a}
\[
0011111111111111101110000111100001 \\
0101000101101110010000110110111101 \\
1000110111011101101011100111101111 \\
1001001000100010000000000000000000 \\
\]
\subsection{Part b}
Our crossover operator will take in 2 chromosomes and produce 2 children. A number, $$n$$,  in the range [0, 35] is selected randomly. The first child has the first $$n$$ bits of the first parents and the last $$36-n$$ bits of the second parent. The second child has the first $$n$$ bits of the second parent and the last $$36-n$$ bits of the second parent.

\[
0011 11111111111111 1000 01101101111011 \\
0101 00010110111001 0111 00001111000011 \\
1000 00100010001001 0000 00000000000001 \\
1001 11011101110111 0101 11001111011111 \\
\]
\subsection{Part c}
Our mutation operator takes in 1 chromosome and produces 1 child. Assume a mutation threshold $$r$$ in the range [0,1]. For every bit in the chromosome, generate a random probability, $$p$$, in the range [0,1]. If $$p > r$$, then flip the bit.
\[
0001 11111110111110 0111 00000101000010 \\
0101 00000110101001 1000 01101011111010 \\
1001 10111101110111 0101 11001011011110 \\
1001 00100011001001 0001 00000000110001 \\
\]
\subsubsection{Decimal Real Number Floating Point Encoding}
\subsection{Part a}
\[
0011 11111111111111 0111 00001111000011 \\
0101 00010110111001 1000 01101101111011 \\
1000 11011101110111 0101 11001111011111 \\
1001 00100010001001 0000 00000000000001 \\
\]
\subsection{Part b}
Our crossover operator will take in 2 chromosomes and produce 2 children. A number, $$n$$,  in the range [0, 35] is selected randomly. The first child has the first $$n$$ bits of the first parents and the last $$36-n$$ bits of the second parent. The second child has the first $$n$$ bits of the second parent and the last $$36-n$$ bits of the second parent.

\[
0011 11111111111111 1000 01101101111011 \\
0101 00010110111001 0111 00001111000011 \\
1000 00100010001001 0000 00000000000001 \\
1001 11011101110111 0101 11001111011111 \\
\]
\subsection{Part c}
Our mutation operator takes in 1 chromosome and produces 1 child. Assume a mutation threshold $$r$$ in the range [0,1]. For every bit in the chromosome, generate a random probability, $$p$$, in the range [0,1]. If $$p > r$$, then flip the bit.
\[
0001 11111110111110 0111 00000101000010 \\
0101 00000110101001 1000 01101011111010 \\
1001 10111101110111 0101 11001011011110 \\
1001 00100011001001 0001 00000000110001 \\
\]

\section{Question 2}
\subsection{Part a}

\subsection{Part b}
The offspring that result from a 1-point crossover are $11010yxxyyyxyxxy$ and $yxyyx01100101101$ since the parents are split at the crossover point, and the ends of the chromosomes are swapped.  This causes one child to start with parent 1's genetic material and end with parent 2's, while the other child starts with parent 2's genetic material and ends with parent 1's.

\subsection{Part c}
Uniform crossover is performed by swapping the genes underlined in the first parent, which yields children 01000101 01111000 01111010 and 10100100 10011001 01101000

\subsection{Part d}

\subsection{Part e}

\subsection{Part f}

\end{document}